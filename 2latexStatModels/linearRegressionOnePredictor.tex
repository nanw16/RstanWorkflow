\documentclass[12pt]{article}

\usepackage[top=1in, bottom=1in, left=1in, right=1in]{geometry}
\usepackage{amsfonts}
\usepackage{amsmath}
\usepackage{amssymb}

\setlength\parindent{0pt}
\begin{document}

\title{Probability Model For A One Predictor Linear Regression Model}
\author{Nan Wu \\ nanw@udel.edu}
\date{}
\maketitle

\section{Linear Regression Model}

The linear regression model is given as
\begin{equation}
y=\alpha+x\beta+\epsilon
\end{equation}

where $y\in \mathbb{R}^N$, $x\in \mathbb{R}^N$, $\alpha \in \mathbb{R}$, $\beta \in \mathbb{R}$, $\epsilon \in \mathbb{R}^N$, and $\epsilon \sim \mathrm{Normal}\left( 0,\sigma \right)$.

In this model, $x$ is the predictor, $y$ is the outcome, $\alpha$ is the intercept, $\beta$ is the slope coefficient, $\epsilon$ is the residual.

\section{Probability Model}

Stan allows us to use improper priors if we don't have any prior knowledge about the parameters. We can therefore start with a simple model by assuming improper priors for $\alpha$, $\beta$ and $\sigma$:
\begin{align*}
\alpha &\sim \mathrm{Uniform}\left( -\infty,\infty \right) \\
\beta &\sim \mathrm{Uniform}\left( -\infty,\infty \right) \\
\sigma &\sim \mathrm{Uniform}\left( 0,\infty \right)
\end{align*}

Putting it all together, the probability model for the single predictor linear regression model is:
\begin{align*}
y_n &\sim \mathrm{Normal}\left( \mu_n, \sigma \right) \\
\mu_n &= \alpha + x_n\beta \\
\alpha &\sim \mathrm{Uniform}\left( -\infty,\infty \right) \\
\beta &\sim \mathrm{Uniform}\left( -\infty,\infty \right) \\
\sigma &\sim \mathrm{Uniform}\left( 0,\infty \right)
\end{align*}

\end{document}