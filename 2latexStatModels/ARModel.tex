\documentclass[12pt]{article}

\usepackage[top=1in, bottom=1in, left=1in, right=1in]{geometry}
\usepackage{amsfonts}
\usepackage{amsmath}
\usepackage{amssymb}

\setlength\parindent{0pt}
\begin{document}

\title{Probability Model For An Autoregressive Model}
\author{Nan Wu \\ nanw@udel.edu}
\date{}
\maketitle

\section{Autoregressive Model}

Let $AR(K)$ denote an autoregressive model of order $K$, which is defined as
\begin{equation}
y_n = \alpha + \sum \limits_{k=1}^K \beta_k y_{n-k} + \epsilon_n
\end{equation}

where $\alpha$ is a constant, $\beta_1, \ldots, \beta_K$ are the parameters of the $AR(K)$ model, and $\{\epsilon_n\}$ is assumed to be a white noise series with mean $0$ and variance $\sigma^2$.

\section{Probability Model}

Here we use an improper prior for $\alpha, \beta, and \sigma$ to illustrate the idea. A weakly informative prior or informative prior could be added if more knowledge of the parameters is available.

The probability model for this $AR(K)$ model is described as:
\begin{align*}
y_n &\sim \mathrm{Normal}\left( \mu_n, \sigma \right) \\
\mu_n &= \alpha + \sum \limits_{k=1}^K \beta_k y_{n-k} \\
\alpha &\sim \mathrm{Uniform}\left( -\infty,\infty \right) \\
\beta &\sim \mathrm{Uniform}\left( -\infty,\infty \right) \\
\sigma &\sim \mathrm{Uniform}\left( 0,\infty \right)
\end{align*}

\end{document}